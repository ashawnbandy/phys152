
We assembled the circuit as described in the lab manual with a function generator , a variable capacitance box, a variable resistance box and to both channels on a 2-channel oscilloscope.   We began by setting the capacitance box to $0.01 \mu F$ and the resistance box to $10 k\Omega$.  We began with the function generator set to produce square waves at 600 Hz and the oscilloscope with the settings described in the lab manual.  This produced the a display on the oscilloscope as seen on Data Sheet \#1 part D.  Next we altered the values for the resistor and capacitance and found that for a given RC, the resulting display was the same without respect to the individual settings for R or for C.  We then set about measuring the time constant of an RC circuit by taking measurements from the oscilloscope.  We adjusted the display so that the height of one cycle filled the 8 divisions.  The time constant is the interval it takes for the voltage to fall to about 37\% of its maximum or where it crosses the 3 vertical division from the bottom.  We estimate that this occurred at 1.2 horizontal divisions and the TIME/DIV was set to 20 microseconds so the time constant was 24 microseconds.\\

As with all of these experiments error in our measurements occurred in three categories.  The first type is error in actually taking the measurements.  For example, we could only reasonably estimate the a given point on the oscilloscope to within one half tick marks.  Second, given the number of leads and the age of the capacitor and resistor boxes, there could be noise or false measurements that we cannot anticipate in our measurements.  Third, we make assumptions to simplify the physics involved in the circuits:  for example, we do not measure the resistance of the leads.\\