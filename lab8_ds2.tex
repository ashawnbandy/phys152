\subsection{DATA SHEET\#2}
	\item Watch the images on the screen as you do the following.  Keep C = 0.01 $\mu F$ on the capacitance box.  From the resistance decade box setting of the R = 10 $k \Omega$, rotate the knob to 20 $k \Omega$, 30 $k \Omega$, up to 50 $k\Omega$.  Describe what happens to the image on the screen.\\
	
	The vertical range for the charging curve contracts, pulling away from both the upper and lower square wave marks.\\
	
	\item At the R = 50 $k\Omega$ and C = $0.01 \mu F$ setting, what is the time constant?\\
	
	0.0005 seconds.
	
	\item Set R back to R = 10 $k \Omega$.  Now dial capacitance up to C = 0.05 $\mu F$  Describe what happens to the image on the screen.\\
	
	The vertical range for the charging curve contracts, pulling away from both the upper and lower square wave marks.\\
	
	\item At the R = 10 $k \Omega$ and C = 0.05 $\mu $ what is the time constant? \\
	
	0.0005 seconds
	
	\item Experimentally measure the time constant of an RC circuit using the oscilloscope.\\
		\begin{enumerate}[1.]
			\item Choose to set the capacitance to some value C between 0.02 $\mu F$ and 0.01 $\mu F$.\\
			
			
			Record the value of $C \pm 5\%$ uncertainty = $ 0.021 \mu F \pm 0.00105 \mu F$\\
			
			\item  Choose to set the resistance to some value R between 2 $k \Omega$ and 9 $k \Omega$\\
			
			Record the value of $R \pm 1 \%$ uncertainty = $ 5 \pm  0.05 k \Omega$\\
			
			\item Compute the time constant of this combination, appropriately propagating the uncertainties.  In order to come out in units of seconds, you have to multiply ohms x farads.\\
			
			RC = $ 0.000105 \pm  0.050011 seconds$\\
			
			\item Because the mathematical description of a discharging C is a little simpler, namely $ \Delta V_c(t) = \Delta V_{batt}(e^{-\frac{t}{RC}})$, set the oscilloscope to trigger on the negative slope of the discharge cycle, and then measure on the screen the time that a discharging capacitor will take to reach 37\% of its initial voltage.  The next steps will indicate how to set the trigger to negative slope in order to observe the char charge cycle first.